\chapter{Conclusione}

\section{Considerazioni}
Questo progetto mi ha fatto entrare nel vivo del lavoro di programmazione e mi ha permesso di andare oltre lo studio astratto della tecnologia Android, mettendomi nella condizione di relazionarmi, dal punto di vista applicativo, con nuovi linguaggi quali ASP.Net per la gestione dei controller lato server. 
Il mio background accademico, nello specifico la conoscenza del paradigma REST, ha fornito la base astratta che mi ha permesso di giungere, nella sfera pratica, alla risoluzione di problemi in merito alla comunicazione tra l’applicazione ed il server. 
Nello sviluppo dell’applicazione sono stati coinvolti anche gli utenti finali tramite sondaggi relativi all’ UI/UX lungo tutto il periodo di sviluppo: sulla base dei dati emersi è stato fatto sì che venissero presi quegli accorgimenti che, mettendo a proprio agio l’utente, rendono un’applicazione una buona applicazione.
Per una migliore "esperienza utente" (User eXperience - UX) siamo partiti dalle caratteristiche di altri servizi di carsharing e ridesharing presenti sul mercato per poi adattarli e renderli conformi ai gusti, alle esigenze e alle preferenze di un pubblico giovane.
La realizzazione del progetto è avvenuta nei tempi previsti. Ciò è stato reso possibile anche grazie all'aiuto dei tutor che hanno saputo guidarci nello sviluppo grazie alle loro competenze e conoscenze acquisite sul campo, ma soprattutto al lavoro di squadra. La collaborazione con i tutor e la sintonia e la serietà che si è venuta a creare nel gruppo hanno fatto sì che il lavoro finale risultasse completo e curato. 
Le nostre conoscenze tecniche acquisite prima e durante il progetto sommate all’affiatamento venutosi a creare nel gruppo hanno reso questo workshop una bellissima esperienza che, tornando indietro, rifarei molto volentieri.
Al termine del percorso sono stati raggiunti gli obiettivi prefissati, sono state acquisite le competenze ed abilità di base per lo sviluppo di un’applicazione, dalla sua fase embrionale di idea al rilascio al pubblico con relativi feedback positivi da parte dell’utenza. 
Si è imparato a lavorare in gruppo condividendo idee, ma soprattutto aiutandosi a vicenda per raggiungere un obbiettivo comune e superare le difficoltà.

\section{Progetto finale}
Tutto il codice su cui si è lavorato durante i tre mesi di progetto sono stati catalogati sulla piattaforma GitHub e il progetto completo è visualizzabile nella relativa repository.
\begin{itemize}
\item \textbf{Codice sorgente client:} \url{https://github.com/movers-gcw/movemate_android}
\item \textbf{Codice sorgente server ASP.Net:} \url{https://github.com/movers-gcw/movemate_api}
\item \textbf{Sito web progetto:} \url{https://movers-gcw.github.io}
\end{itemize}
