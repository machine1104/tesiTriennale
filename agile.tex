\chapter{Metodologia Agile}
Nell'ingegneria del software, la metodologia agile si riferisce a un insieme di metodi di sviluppo del software derivati dai principi del "Manifesto per lo sviluppo agile del software" pubblicato nel 2001. I metodi agili propongono un approccio focalizzato sull'obiettivo di consegnare software funzionante e di qualità al cliente  in tempi brevi e frequentemente.
La metodologia agile è alla base del progetto descritto in seguito e del lavoro svolto durante tutto il periodo di sviluppo.

\section{Introduzione}
Fra le pratiche promosse dai metodi agili ci sono lo sviluppo iterativo e incrementale, la pianificazione adattiva, il coinvolgimento diretto e continuo del cliente nel processo di sviluppo e la formazione di piccoli team di lavoro.
La maggior parte dei metodi cosiddetti agili sviluppa il software in finestre di tempo limitate chiamate iterazioni che, in genere, durano qualche settimana. Nel caso di questo workshop le iterazioni erano rappresentate da tre "milestones" in cui i team dovevano presentare il lavoro svolto fino a quel momento. Ogni iterazione è un piccolo progetto indipendente e contiene tutto ciò che è necessario per rilasciare un piccolo incremento nelle funzionalità del software: planificazione (planning), analisi dei requisiti, progettazione, implementazione, test e documentazione. Alla fine di ogni iterazione il team deve rivalutare le priorità del progetto.
I metodi agili preferiscono la comunicazione in tempo reale a quella scritta.
Il team agile è composto da tutte le persone necessarie per terminare il progetto e includono anche il cliente.
L'obiettivo è la piena soddisfazione del cliente e non solo l'adempimento di un contratto. Le metodologie agile, inoltre, possono consentire all'abbattimento dei costi e dei tempi di sviluppo aumentando la qualità del prodotto finale.

\section{Principi}
I principi su cui si basa una metodologia agile sono quattro:
\begin{itemize}
\item le relazioni e la comunicazione tra gli attori di un progetto sono più importanti degli strumenti utilizzati;
\item è più importante avere software funzionante che documentazione;
\item bisogna collaborare con i clienti poichè la collaborazione diretta offre risultati migliori dei rapporti puramente contrattuali;
\item bisogna accettare il cambiamento e il team deve essere pronto a modificare le priorità di lavoro.
\end{itemize}

\section{Pratiche utilizzate}
Le singole pratiche applicabili all'interno di una metodologia agile sono molteplici e dipendono dalle necessità e dall'approccio scelto. Ogni pratica ha pro e contro. Ad esempio, in Extreme Programming (uno dei tanti metodi esistenti), la mancanza assoluta di qualsiasi forma di progettazione e documentazione è compensata con lo strettissimo coinvolgimento del cliente nello sviluppo e con la programmazione in coppia.
Sono elencate ore le pratiche utilizzate nel corso del progetto.

\begin{itemize}
\item \textbf{Coinvolgimento del cliente - }Vi sono differenti gradi di coinvolgimento (come detto precedentemente per il caso di Extreme Programming). In questo caso specifico il cliente è stato ascoltato in più occasioni: per la convalida dell'idea e per i successivi test sull'esperienza d'uso.
\item \textbf{Cultura di Team - }Fondamentale nel seguire approcci agili è la collaborazione e l'approccio mentale e pratico del team di sviluppo stesso. Ci si orienta verso un modus operandi "di gruppo" che andrà a premiare (o viceversa) il gruppo unicamente sulla base del raggiungimento degli obiettivi di team previsti per quel dato intervallo temporale.
\item \textbf{Consegne frequenti - }Le consegne frequenti garantiscono di continuare il lavoro da un codice già scritto nella consegna precedente , di  offrire al cliente una parte di progetto distraendolo fino alla consegna finale, di effettuare test su porzioni di applicazione e ricevere feedback su eventuali nuove funzionalità da inserire o correggere.
Le iterazioni sono rappresentate dalle tre milestones del workshop.
\item \textbf{Progettazione e documentazione - }Pensare che le metodologie leggere eliminino la progettazione e la documentazione è un errore. Infatti è sempre buona norma garantire e non trascurare la parte di progettazione e documentazione. In questo caso sono stati disegnati i diagrammi UML delle classi e dei casi d'uso.
\end{itemize}