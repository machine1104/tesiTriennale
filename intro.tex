\begin{abstract}
Durante il corso "Reti di calcolatori", tenuto dal professore Andrea Vitaletti, è stato proposto agli studenti di partecipare ad un laboratorio tenuto da Google presso l'Università La Sapienza relativo allo sviluppo cloud e web di servizi, nel corso del quale sono stati trattati argomenti relativi alla metodologia di sviluppo Agile e allo sviluppo di una corretta User Interface e relativa User Experience. L’obbiettivo era quello di fornire ai partecipanti le conoscenze necessarie per dare vita ad un’applicazione (web o nativa) seguendo le linee guida presentate durante le varie giornate del workshop stesso. 
Nel dicembre del 2016 il team di cui facevo parte ha iniziato a sviluppare l’applicazione: MoveMate. L’idea era di creare una piattaforma che consentisse agli studenti universitari romani (e in seguito di tutta Italia) di mettersi in contatto tra di loro così che potessero condividere il tragitto (e relative spese) verso (e da) le sedi universitarie. 
Partendo dall’idea del prototipo, attraverso tre milestones, ad aprile 2017 abbiamo portato a termine il lavoro con il rilascio al pubblico della versione finale dell'applicazione.
\end{abstract}